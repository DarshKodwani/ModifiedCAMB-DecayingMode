\documentclass[12pt]{article}

\usepackage[right=.75in, left=.75in, top=1in, bottom = 1 in]{geometry}

\usepackage{mathptmx}
\usepackage{times}

\usepackage{mathtools}

\usepackage{float}
\usepackage{graphicx,epsfig}
\pagestyle{plain}
\usepackage{amsmath}
\usepackage{amssymb}
\usepackage{cite}
\usepackage{color,colordvi}
\usepackage{cancel}

\usepackage{caption}
\usepackage{subcaption}

\usepackage{array, multirow}

\newcommand{\RN}[1]{%
  \textup{\uppercase\expandafter{\romannumeral#1}}%
}


\newcommand\myprop{\stackrel{\propto}{\mathclap{\scriptsize\mbox{GRF}}}}


\usepackage[]{hyperref}
\usepackage{xcolor}
\hypersetup{
    colorlinks,
    linkcolor={black!50!black},
    citecolor={blue!50!black},
    urlcolor={blue!80!black}
}

\numberwithin{equation}{section}
\title{Summary of decaying mode}
\author{--}
\date{ \today}

\begin{document}

\setlength{\belowcaptionskip}{2.0 pt}
\setlength{\intextsep}{3.0 pt}

\maketitle

\tableofcontents

\section{Introduction to decaying mode}

%Write the equations for the decaying mode in Synchronus gauge. 
%\\
%Comparing the Cl's when $\omega$ is removed.
%\begin{figure}[h!]
%\begin{center}
%\includegraphics[scale = 0.4]{Comparing_omtau.png}
%\vspace{5mm}
%\caption{Comparing the Cl's when $\omega$ is removed. }
%\label{fig:1}
%\end{center}
%\end{figure}

\section{Low $l$ divergence}

We look at the divergence at low $l$ in CAMB and CLASS.

\begin{figure}[ht]\centering
        \begin{subfigure}{.5\textwidth}
               \includegraphics[scale=0.2]{Decaying_mode_TE.png}
               \caption{TE Cl's}
                \label{fig:P1} 
        \end{subfigure}%
        \begin{subfigure}{.5\textwidth}
                \includegraphics[scale=0.2]{Decaying_mode_EE.png}
                \caption{EE Cl's}
                \label{fig:P2}
        \end{subfigure}
	\begin{subfigure}{0.5\textwidth}
                \includegraphics[scale=0.3]{Decaying_mode_TT.png}
                \caption{TT Cl's}
                \label{fig:P3}
        \end{subfigure}
       \caption{Decaying mode figures with $n_s = 0.96$} \label{fig:growing mode}
\end{figure}

\newpage 

We now look at different spectral indices that are used in \cite{Amendola:2004rt} in CAMB. The "Model" Cl's are for the canonical cosmology

\begin{figure}[ht]\centering
        \begin{subfigure}{.5\textwidth}
               \includegraphics[scale=0.2]{tilt_1.pdf}
               \caption{TT decaying mode for tilt $n_s =1$}
                \label{fig:P1} 
        \end{subfigure}%
        \begin{subfigure}{.5\textwidth}
                \includegraphics[scale=0.2]{tilt_3.pdf}
                \caption{TT decaying mode for tilt $n_s =3$}
                \label{fig:P2}
        \end{subfigure}
       \caption{Decaying mode figures different tilts} \label{fig:growing mode}
\end{figure}

We now present the same spectral indices as in \cite{Amendola:2004rt} in CLASS. As Xin showed in class, the low l divergence goes away for $n_s = 4$, i.e the index is $k^3$. 
This suggests, as expected, that there are some rogue factors of $k$ in the denominator for the decaying mode.
There are also some differences in the decaying mode plots in class and camb as seen in figure \ref{cvsc}. It might just be a difference in normalization (Multipling the class Cl's by 4 makes them look similar at high l) but I am not sure yet. 

\begin{figure}[h!]
\begin{center}
\includegraphics[scale = 0.4]{class_camb_decay_tilt1.pdf}
\vspace{5mm}
\caption{Comparing the Cl's for the decaying mode for $n_s=1$ in class and camb along with the model Cl's. }
\label{cvsc}
\end{center}
\end{figure}

\section{Solutions attempted}

\subsection{Change of gauge}

Changing the gauge from Synchronus (as is used in CAMB) to Newtonian in CLASS we see no real difference in the Cl's.

\begin{figure}[h!]
\begin{center}
\includegraphics[scale = 0.6]{NvsS.pdf}
\vspace{5mm}
\caption{Cl's in Newtonian and Synchronus gauge.}
\label{NvsS}
\end{center}
\end{figure}

\newpage

\subsection{Try to normalize the transfer function on sub horizon scales}

Various attempts have been made to normalize the decaying mode. In particular we have tried to use the high k transfer functions to find a function of ell to normalize the Cl's by. It hasn't been successfull yet. 

\begin{figure}[h!]
\begin{center}
\includegraphics[scale = 0.4]{transfer_decayls.png}
\vspace{5mm}
\caption{Transfer function for the decaying mode for various ells.}
\label{transfer_decay}
\end{center}
\end{figure}

As a comparison, we have the growing mode transfer function:

\begin{figure}[h!]
\begin{center}
\includegraphics[scale = 0.5]{transfer_growingls.png}
\vspace{5mm}
\caption{Transfer function for the growing mode for various ells.}
\label{transfer_grow}
\end{center}
\end{figure}

\newpage

\subsection{Change accuracy settings}

Changing the accuracy settings does not effect the divergence at low $l$. 

\begin{figure}[ht]\centering
        \begin{subfigure}{.5\textwidth}
               \includegraphics[scale=0.4]{time_sampling.png}
               \caption{TE Cl's}
                \label{tsamp} 
        \end{subfigure}%
        \begin{subfigure}{.5\textwidth}
                \includegraphics[scale=0.4]{hierarchy_comparison.png}
                \caption{EE Cl's}
                \label{hsamp}
        \end{subfigure}
	\begin{subfigure}{0.5\textwidth}
                \includegraphics[scale=0.3]{interpolation_comparison.png}
                \caption{TT Cl's}
                \label{isamp}
        \end{subfigure}
       \caption{Checking various accuracy settings in CAMB} \label{accuracy}
\end{figure}


\subsection{The transfer functions for different redshifts}

Currently working on it - Daan provided first plots. 

\newpage
\subsection{Radiation dominated universe}

We see that the photon density $\Delta_\gamma$ is effected by the decaying mode as seen in the density evolution plots. In particular the initial amplitude is much higher for the decaying mode. 

 \begin{figure}[ht]\centering
        \begin{subfigure}{.5\textwidth}
               \includegraphics[scale=0.22]{background_desnities_kp02_decayingmode.png}
               \caption{Decaying mode for $k=0.02$}
                \label{gamma_d} 
        \end{subfigure}%
        \begin{subfigure}{.5\textwidth}
                \includegraphics[scale=0.25]{background_desnities_kp02_growingmode.png}
                \caption{Growing mode $k=0.02$}
                \label{gamma_g}
        \end{subfigure}
       \caption{$\Delta_\gamma$ evolution for decaying and growing modes} \label{gamma_sampling}
\end{figure}

Furthermore we can look at the radiation dominated universe by increasing $N_{eff}$. 

\begin{figure}[h!]
\begin{center}
\includegraphics[scale = 0.5]{neff.png}
\vspace{5mm}
\caption{Cl's for a range of $N_{eff}$.}
\label{Neff}
\end{center}
\end{figure}

Doesn't effect the low l divergence. 
\\
\\
Surprisingly it appears that the decaying and growing modes are in phase for $\Delta_\gamma$. One would expect the phase of the oscillation to be apparent in these modes. 

\begin{figure}[h!]
\begin{center}
\includegraphics[scale = 0.5]{gamma_comp01.pdf}
\vspace{5mm}
\caption{Comparing the evolution of $\Delta_\gamma$ for decaying and growing modes for $k = 0.1 \ Mpc^{-1}$.}
\label{gamma_comp} 
\end{center}
\end{figure}

\subsection{More to follow...}
 

\begin{thebibliography}{99}

%\cite{Amendola:2004rt}
\bibitem{Amendola:2004rt}
  L.~Amendola and F.~Finelli,
  %``On the effects due to a decaying cosmological fluctuation,''
  Phys.\ Rev.\ Lett.\  {\bf 94} (2005) 221303
  doi:10.1103/PhysRevLett.94.221303
  [astro-ph/0411273].
  %%CITATION = doi:10.1103/PhysRevLett.94.221303;%%
  %8 citations counted in INSPIRE as of 14 Sep 2017

	

\end{thebibliography}


















\end{document}